
% Default to the notebook output style

    


% Inherit from the specified cell style.




    
\documentclass[11pt]{article}

    
    
    \usepackage[T1]{fontenc}
    % Nicer default font (+ math font) than Computer Modern for most use cases
    \usepackage{mathpazo}

    % Basic figure setup, for now with no caption control since it's done
    % automatically by Pandoc (which extracts ![](path) syntax from Markdown).
    \usepackage{graphicx}
    % We will generate all images so they have a width \maxwidth. This means
    % that they will get their normal width if they fit onto the page, but
    % are scaled down if they would overflow the margins.
    \makeatletter
    \def\maxwidth{\ifdim\Gin@nat@width>\linewidth\linewidth
    \else\Gin@nat@width\fi}
    \makeatother
    \let\Oldincludegraphics\includegraphics
    % Set max figure width to be 80% of text width, for now hardcoded.
    \renewcommand{\includegraphics}[1]{\Oldincludegraphics[width=.8\maxwidth]{#1}}
    % Ensure that by default, figures have no caption (until we provide a
    % proper Figure object with a Caption API and a way to capture that
    % in the conversion process - todo).
    \usepackage{caption}
    \DeclareCaptionLabelFormat{nolabel}{}
    \captionsetup{labelformat=nolabel}

    \usepackage{adjustbox} % Used to constrain images to a maximum size 
    \usepackage{xcolor} % Allow colors to be defined
    \usepackage{enumerate} % Needed for markdown enumerations to work
    \usepackage{geometry} % Used to adjust the document margins
    \usepackage{amsmath} % Equations
    \usepackage{amssymb} % Equations
    \usepackage{textcomp} % defines textquotesingle
    % Hack from http://tex.stackexchange.com/a/47451/13684:
    \AtBeginDocument{%
        \def\PYZsq{\textquotesingle}% Upright quotes in Pygmentized code
    }
    \usepackage{upquote} % Upright quotes for verbatim code
    \usepackage{eurosym} % defines \euro
    \usepackage[mathletters]{ucs} % Extended unicode (utf-8) support
    \usepackage[utf8x]{inputenc} % Allow utf-8 characters in the tex document
    \usepackage{fancyvrb} % verbatim replacement that allows latex
    \usepackage{grffile} % extends the file name processing of package graphics 
                         % to support a larger range 
    % The hyperref package gives us a pdf with properly built
    % internal navigation ('pdf bookmarks' for the table of contents,
    % internal cross-reference links, web links for URLs, etc.)
    \usepackage{hyperref}
    \usepackage{longtable} % longtable support required by pandoc >1.10
    \usepackage{booktabs}  % table support for pandoc > 1.12.2
    \usepackage[inline]{enumitem} % IRkernel/repr support (it uses the enumerate* environment)
    \usepackage[normalem]{ulem} % ulem is needed to support strikethroughs (\sout)
                                % normalem makes italics be italics, not underlines
    \usepackage{mathrsfs}
    

    
    
    % Colors for the hyperref package
    \definecolor{urlcolor}{rgb}{0,.145,.698}
    \definecolor{linkcolor}{rgb}{.71,0.21,0.01}
    \definecolor{citecolor}{rgb}{.12,.54,.11}

    % ANSI colors
    \definecolor{ansi-black}{HTML}{3E424D}
    \definecolor{ansi-black-intense}{HTML}{282C36}
    \definecolor{ansi-red}{HTML}{E75C58}
    \definecolor{ansi-red-intense}{HTML}{B22B31}
    \definecolor{ansi-green}{HTML}{00A250}
    \definecolor{ansi-green-intense}{HTML}{007427}
    \definecolor{ansi-yellow}{HTML}{DDB62B}
    \definecolor{ansi-yellow-intense}{HTML}{B27D12}
    \definecolor{ansi-blue}{HTML}{208FFB}
    \definecolor{ansi-blue-intense}{HTML}{0065CA}
    \definecolor{ansi-magenta}{HTML}{D160C4}
    \definecolor{ansi-magenta-intense}{HTML}{A03196}
    \definecolor{ansi-cyan}{HTML}{60C6C8}
    \definecolor{ansi-cyan-intense}{HTML}{258F8F}
    \definecolor{ansi-white}{HTML}{C5C1B4}
    \definecolor{ansi-white-intense}{HTML}{A1A6B2}
    \definecolor{ansi-default-inverse-fg}{HTML}{FFFFFF}
    \definecolor{ansi-default-inverse-bg}{HTML}{000000}

    % commands and environments needed by pandoc snippets
    % extracted from the output of `pandoc -s`
    \providecommand{\tightlist}{%
      \setlength{\itemsep}{0pt}\setlength{\parskip}{0pt}}
    \DefineVerbatimEnvironment{Highlighting}{Verbatim}{commandchars=\\\{\}}
    % Add ',fontsize=\small' for more characters per line
    \newenvironment{Shaded}{}{}
    \newcommand{\KeywordTok}[1]{\textcolor[rgb]{0.00,0.44,0.13}{\textbf{{#1}}}}
    \newcommand{\DataTypeTok}[1]{\textcolor[rgb]{0.56,0.13,0.00}{{#1}}}
    \newcommand{\DecValTok}[1]{\textcolor[rgb]{0.25,0.63,0.44}{{#1}}}
    \newcommand{\BaseNTok}[1]{\textcolor[rgb]{0.25,0.63,0.44}{{#1}}}
    \newcommand{\FloatTok}[1]{\textcolor[rgb]{0.25,0.63,0.44}{{#1}}}
    \newcommand{\CharTok}[1]{\textcolor[rgb]{0.25,0.44,0.63}{{#1}}}
    \newcommand{\StringTok}[1]{\textcolor[rgb]{0.25,0.44,0.63}{{#1}}}
    \newcommand{\CommentTok}[1]{\textcolor[rgb]{0.38,0.63,0.69}{\textit{{#1}}}}
    \newcommand{\OtherTok}[1]{\textcolor[rgb]{0.00,0.44,0.13}{{#1}}}
    \newcommand{\AlertTok}[1]{\textcolor[rgb]{1.00,0.00,0.00}{\textbf{{#1}}}}
    \newcommand{\FunctionTok}[1]{\textcolor[rgb]{0.02,0.16,0.49}{{#1}}}
    \newcommand{\RegionMarkerTok}[1]{{#1}}
    \newcommand{\ErrorTok}[1]{\textcolor[rgb]{1.00,0.00,0.00}{\textbf{{#1}}}}
    \newcommand{\NormalTok}[1]{{#1}}
    
    % Additional commands for more recent versions of Pandoc
    \newcommand{\ConstantTok}[1]{\textcolor[rgb]{0.53,0.00,0.00}{{#1}}}
    \newcommand{\SpecialCharTok}[1]{\textcolor[rgb]{0.25,0.44,0.63}{{#1}}}
    \newcommand{\VerbatimStringTok}[1]{\textcolor[rgb]{0.25,0.44,0.63}{{#1}}}
    \newcommand{\SpecialStringTok}[1]{\textcolor[rgb]{0.73,0.40,0.53}{{#1}}}
    \newcommand{\ImportTok}[1]{{#1}}
    \newcommand{\DocumentationTok}[1]{\textcolor[rgb]{0.73,0.13,0.13}{\textit{{#1}}}}
    \newcommand{\AnnotationTok}[1]{\textcolor[rgb]{0.38,0.63,0.69}{\textbf{\textit{{#1}}}}}
    \newcommand{\CommentVarTok}[1]{\textcolor[rgb]{0.38,0.63,0.69}{\textbf{\textit{{#1}}}}}
    \newcommand{\VariableTok}[1]{\textcolor[rgb]{0.10,0.09,0.49}{{#1}}}
    \newcommand{\ControlFlowTok}[1]{\textcolor[rgb]{0.00,0.44,0.13}{\textbf{{#1}}}}
    \newcommand{\OperatorTok}[1]{\textcolor[rgb]{0.40,0.40,0.40}{{#1}}}
    \newcommand{\BuiltInTok}[1]{{#1}}
    \newcommand{\ExtensionTok}[1]{{#1}}
    \newcommand{\PreprocessorTok}[1]{\textcolor[rgb]{0.74,0.48,0.00}{{#1}}}
    \newcommand{\AttributeTok}[1]{\textcolor[rgb]{0.49,0.56,0.16}{{#1}}}
    \newcommand{\InformationTok}[1]{\textcolor[rgb]{0.38,0.63,0.69}{\textbf{\textit{{#1}}}}}
    \newcommand{\WarningTok}[1]{\textcolor[rgb]{0.38,0.63,0.69}{\textbf{\textit{{#1}}}}}
    
    
    % Define a nice break command that doesn't care if a line doesn't already
    % exist.
    \def\br{\hspace*{\fill} \\* }
    % Math Jax compatibility definitions
    \def\gt{>}
    \def\lt{<}
    \let\Oldtex\TeX
    \let\Oldlatex\LaTeX
    \renewcommand{\TeX}{\textrm{\Oldtex}}
    \renewcommand{\LaTeX}{\textrm{\Oldlatex}}
    % Document parameters
    % Document title
    \title{MIEI/MI - Estruturas Criptográficas\\
            \large Trabalho Prático 0}
    
    \author{
        João Alves\\
        \texttt{a77070@alunos.uminho.pt}
        \and
        Nuno Leite\\
        \texttt{a70132@alunos.uminho.pt}
      }
    \date{Universidade do Minho}
    
    
    
    

    % Pygments definitions
    
\makeatletter
\def\PY@reset{\let\PY@it=\relax \let\PY@bf=\relax%
    \let\PY@ul=\relax \let\PY@tc=\relax%
    \let\PY@bc=\relax \let\PY@ff=\relax}
\def\PY@tok#1{\csname PY@tok@#1\endcsname}
\def\PY@toks#1+{\ifx\relax#1\empty\else%
    \PY@tok{#1}\expandafter\PY@toks\fi}
\def\PY@do#1{\PY@bc{\PY@tc{\PY@ul{%
    \PY@it{\PY@bf{\PY@ff{#1}}}}}}}
\def\PY#1#2{\PY@reset\PY@toks#1+\relax+\PY@do{#2}}

\expandafter\def\csname PY@tok@w\endcsname{\def\PY@tc##1{\textcolor[rgb]{0.73,0.73,0.73}{##1}}}
\expandafter\def\csname PY@tok@c\endcsname{\let\PY@it=\textit\def\PY@tc##1{\textcolor[rgb]{0.25,0.50,0.50}{##1}}}
\expandafter\def\csname PY@tok@cp\endcsname{\def\PY@tc##1{\textcolor[rgb]{0.74,0.48,0.00}{##1}}}
\expandafter\def\csname PY@tok@k\endcsname{\let\PY@bf=\textbf\def\PY@tc##1{\textcolor[rgb]{0.00,0.50,0.00}{##1}}}
\expandafter\def\csname PY@tok@kp\endcsname{\def\PY@tc##1{\textcolor[rgb]{0.00,0.50,0.00}{##1}}}
\expandafter\def\csname PY@tok@kt\endcsname{\def\PY@tc##1{\textcolor[rgb]{0.69,0.00,0.25}{##1}}}
\expandafter\def\csname PY@tok@o\endcsname{\def\PY@tc##1{\textcolor[rgb]{0.40,0.40,0.40}{##1}}}
\expandafter\def\csname PY@tok@ow\endcsname{\let\PY@bf=\textbf\def\PY@tc##1{\textcolor[rgb]{0.67,0.13,1.00}{##1}}}
\expandafter\def\csname PY@tok@nb\endcsname{\def\PY@tc##1{\textcolor[rgb]{0.00,0.50,0.00}{##1}}}
\expandafter\def\csname PY@tok@nf\endcsname{\def\PY@tc##1{\textcolor[rgb]{0.00,0.00,1.00}{##1}}}
\expandafter\def\csname PY@tok@nc\endcsname{\let\PY@bf=\textbf\def\PY@tc##1{\textcolor[rgb]{0.00,0.00,1.00}{##1}}}
\expandafter\def\csname PY@tok@nn\endcsname{\let\PY@bf=\textbf\def\PY@tc##1{\textcolor[rgb]{0.00,0.00,1.00}{##1}}}
\expandafter\def\csname PY@tok@ne\endcsname{\let\PY@bf=\textbf\def\PY@tc##1{\textcolor[rgb]{0.82,0.25,0.23}{##1}}}
\expandafter\def\csname PY@tok@nv\endcsname{\def\PY@tc##1{\textcolor[rgb]{0.10,0.09,0.49}{##1}}}
\expandafter\def\csname PY@tok@no\endcsname{\def\PY@tc##1{\textcolor[rgb]{0.53,0.00,0.00}{##1}}}
\expandafter\def\csname PY@tok@nl\endcsname{\def\PY@tc##1{\textcolor[rgb]{0.63,0.63,0.00}{##1}}}
\expandafter\def\csname PY@tok@ni\endcsname{\let\PY@bf=\textbf\def\PY@tc##1{\textcolor[rgb]{0.60,0.60,0.60}{##1}}}
\expandafter\def\csname PY@tok@na\endcsname{\def\PY@tc##1{\textcolor[rgb]{0.49,0.56,0.16}{##1}}}
\expandafter\def\csname PY@tok@nt\endcsname{\let\PY@bf=\textbf\def\PY@tc##1{\textcolor[rgb]{0.00,0.50,0.00}{##1}}}
\expandafter\def\csname PY@tok@nd\endcsname{\def\PY@tc##1{\textcolor[rgb]{0.67,0.13,1.00}{##1}}}
\expandafter\def\csname PY@tok@s\endcsname{\def\PY@tc##1{\textcolor[rgb]{0.73,0.13,0.13}{##1}}}
\expandafter\def\csname PY@tok@sd\endcsname{\let\PY@it=\textit\def\PY@tc##1{\textcolor[rgb]{0.73,0.13,0.13}{##1}}}
\expandafter\def\csname PY@tok@si\endcsname{\let\PY@bf=\textbf\def\PY@tc##1{\textcolor[rgb]{0.73,0.40,0.53}{##1}}}
\expandafter\def\csname PY@tok@se\endcsname{\let\PY@bf=\textbf\def\PY@tc##1{\textcolor[rgb]{0.73,0.40,0.13}{##1}}}
\expandafter\def\csname PY@tok@sr\endcsname{\def\PY@tc##1{\textcolor[rgb]{0.73,0.40,0.53}{##1}}}
\expandafter\def\csname PY@tok@ss\endcsname{\def\PY@tc##1{\textcolor[rgb]{0.10,0.09,0.49}{##1}}}
\expandafter\def\csname PY@tok@sx\endcsname{\def\PY@tc##1{\textcolor[rgb]{0.00,0.50,0.00}{##1}}}
\expandafter\def\csname PY@tok@m\endcsname{\def\PY@tc##1{\textcolor[rgb]{0.40,0.40,0.40}{##1}}}
\expandafter\def\csname PY@tok@gh\endcsname{\let\PY@bf=\textbf\def\PY@tc##1{\textcolor[rgb]{0.00,0.00,0.50}{##1}}}
\expandafter\def\csname PY@tok@gu\endcsname{\let\PY@bf=\textbf\def\PY@tc##1{\textcolor[rgb]{0.50,0.00,0.50}{##1}}}
\expandafter\def\csname PY@tok@gd\endcsname{\def\PY@tc##1{\textcolor[rgb]{0.63,0.00,0.00}{##1}}}
\expandafter\def\csname PY@tok@gi\endcsname{\def\PY@tc##1{\textcolor[rgb]{0.00,0.63,0.00}{##1}}}
\expandafter\def\csname PY@tok@gr\endcsname{\def\PY@tc##1{\textcolor[rgb]{1.00,0.00,0.00}{##1}}}
\expandafter\def\csname PY@tok@ge\endcsname{\let\PY@it=\textit}
\expandafter\def\csname PY@tok@gs\endcsname{\let\PY@bf=\textbf}
\expandafter\def\csname PY@tok@gp\endcsname{\let\PY@bf=\textbf\def\PY@tc##1{\textcolor[rgb]{0.00,0.00,0.50}{##1}}}
\expandafter\def\csname PY@tok@go\endcsname{\def\PY@tc##1{\textcolor[rgb]{0.53,0.53,0.53}{##1}}}
\expandafter\def\csname PY@tok@gt\endcsname{\def\PY@tc##1{\textcolor[rgb]{0.00,0.27,0.87}{##1}}}
\expandafter\def\csname PY@tok@err\endcsname{\def\PY@bc##1{\setlength{\fboxsep}{0pt}\fcolorbox[rgb]{1.00,0.00,0.00}{1,1,1}{\strut ##1}}}
\expandafter\def\csname PY@tok@kc\endcsname{\let\PY@bf=\textbf\def\PY@tc##1{\textcolor[rgb]{0.00,0.50,0.00}{##1}}}
\expandafter\def\csname PY@tok@kd\endcsname{\let\PY@bf=\textbf\def\PY@tc##1{\textcolor[rgb]{0.00,0.50,0.00}{##1}}}
\expandafter\def\csname PY@tok@kn\endcsname{\let\PY@bf=\textbf\def\PY@tc##1{\textcolor[rgb]{0.00,0.50,0.00}{##1}}}
\expandafter\def\csname PY@tok@kr\endcsname{\let\PY@bf=\textbf\def\PY@tc##1{\textcolor[rgb]{0.00,0.50,0.00}{##1}}}
\expandafter\def\csname PY@tok@bp\endcsname{\def\PY@tc##1{\textcolor[rgb]{0.00,0.50,0.00}{##1}}}
\expandafter\def\csname PY@tok@fm\endcsname{\def\PY@tc##1{\textcolor[rgb]{0.00,0.00,1.00}{##1}}}
\expandafter\def\csname PY@tok@vc\endcsname{\def\PY@tc##1{\textcolor[rgb]{0.10,0.09,0.49}{##1}}}
\expandafter\def\csname PY@tok@vg\endcsname{\def\PY@tc##1{\textcolor[rgb]{0.10,0.09,0.49}{##1}}}
\expandafter\def\csname PY@tok@vi\endcsname{\def\PY@tc##1{\textcolor[rgb]{0.10,0.09,0.49}{##1}}}
\expandafter\def\csname PY@tok@vm\endcsname{\def\PY@tc##1{\textcolor[rgb]{0.10,0.09,0.49}{##1}}}
\expandafter\def\csname PY@tok@sa\endcsname{\def\PY@tc##1{\textcolor[rgb]{0.73,0.13,0.13}{##1}}}
\expandafter\def\csname PY@tok@sb\endcsname{\def\PY@tc##1{\textcolor[rgb]{0.73,0.13,0.13}{##1}}}
\expandafter\def\csname PY@tok@sc\endcsname{\def\PY@tc##1{\textcolor[rgb]{0.73,0.13,0.13}{##1}}}
\expandafter\def\csname PY@tok@dl\endcsname{\def\PY@tc##1{\textcolor[rgb]{0.73,0.13,0.13}{##1}}}
\expandafter\def\csname PY@tok@s2\endcsname{\def\PY@tc##1{\textcolor[rgb]{0.73,0.13,0.13}{##1}}}
\expandafter\def\csname PY@tok@sh\endcsname{\def\PY@tc##1{\textcolor[rgb]{0.73,0.13,0.13}{##1}}}
\expandafter\def\csname PY@tok@s1\endcsname{\def\PY@tc##1{\textcolor[rgb]{0.73,0.13,0.13}{##1}}}
\expandafter\def\csname PY@tok@mb\endcsname{\def\PY@tc##1{\textcolor[rgb]{0.40,0.40,0.40}{##1}}}
\expandafter\def\csname PY@tok@mf\endcsname{\def\PY@tc##1{\textcolor[rgb]{0.40,0.40,0.40}{##1}}}
\expandafter\def\csname PY@tok@mh\endcsname{\def\PY@tc##1{\textcolor[rgb]{0.40,0.40,0.40}{##1}}}
\expandafter\def\csname PY@tok@mi\endcsname{\def\PY@tc##1{\textcolor[rgb]{0.40,0.40,0.40}{##1}}}
\expandafter\def\csname PY@tok@il\endcsname{\def\PY@tc##1{\textcolor[rgb]{0.40,0.40,0.40}{##1}}}
\expandafter\def\csname PY@tok@mo\endcsname{\def\PY@tc##1{\textcolor[rgb]{0.40,0.40,0.40}{##1}}}
\expandafter\def\csname PY@tok@ch\endcsname{\let\PY@it=\textit\def\PY@tc##1{\textcolor[rgb]{0.25,0.50,0.50}{##1}}}
\expandafter\def\csname PY@tok@cm\endcsname{\let\PY@it=\textit\def\PY@tc##1{\textcolor[rgb]{0.25,0.50,0.50}{##1}}}
\expandafter\def\csname PY@tok@cpf\endcsname{\let\PY@it=\textit\def\PY@tc##1{\textcolor[rgb]{0.25,0.50,0.50}{##1}}}
\expandafter\def\csname PY@tok@c1\endcsname{\let\PY@it=\textit\def\PY@tc##1{\textcolor[rgb]{0.25,0.50,0.50}{##1}}}
\expandafter\def\csname PY@tok@cs\endcsname{\let\PY@it=\textit\def\PY@tc##1{\textcolor[rgb]{0.25,0.50,0.50}{##1}}}

\def\PYZbs{\char`\\}
\def\PYZus{\char`\_}
\def\PYZob{\char`\{}
\def\PYZcb{\char`\}}
\def\PYZca{\char`\^}
\def\PYZam{\char`\&}
\def\PYZlt{\char`\<}
\def\PYZgt{\char`\>}
\def\PYZsh{\char`\#}
\def\PYZpc{\char`\%}
\def\PYZdl{\char`\$}
\def\PYZhy{\char`\-}
\def\PYZsq{\char`\'}
\def\PYZdq{\char`\"}
\def\PYZti{\char`\~}
% for compatibility with earlier versions
\def\PYZat{@}
\def\PYZlb{[}
\def\PYZrb{]}
\makeatother


    % Exact colors from NB
    \definecolor{incolor}{rgb}{0.0, 0.0, 0.5}
    \definecolor{outcolor}{rgb}{0.545, 0.0, 0.0}



    
    % Prevent overflowing lines due to hard-to-break entities
    \sloppy 
    % Setup hyperref package
    \hypersetup{
      breaklinks=true,  % so long urls are correctly broken across lines
      colorlinks=true,
      urlcolor=urlcolor,
      linkcolor=linkcolor,
      citecolor=citecolor,
      }
    % Slightly bigger margins than the latex defaults
    
    \geometry{verbose,tmargin=1in,bmargin=1in,lmargin=1in,rmargin=1in}
    
    

    \begin{document}
    
    
    \maketitle
    
    
    \hypertarget{introduuxe7uxe3o}{%
\section{Introdução}\label{introduuxe7uxe3o}}

    A resolução deste trabalho prático tem como objetivo servir de iniciação
à componente prática da unidade curricular de Estruturas criptográficas,
onde se pretende: - Instalar as ferramentas computacionais necessárias
para a realização dos trabalhos práticos. - Demonstrar pequenas
aplicações implementadas em \textbf{Python} e em \textbf{SageMath}.

A aplicação em \textbf{Python} deve ser implementada de tal forma que
permita a comunicação entre um emissor e um recetor, com as seguintes
caracteristicas: - Criptograma e metadados devem ser autenticados. -
Utilizar uma cifra simétrica em modo \emph{stream cipher}. - Autenticar
previamente a chave.

A aplicação em \textbf{SageMath} deve ser implementada de tal forma que:
- Crie 4 corpos finitos primos. - Crie um \emph{plot} de uma função em
cada um dos corpos finitos primos. - Teste, por amostragem, o facto de
que, considerando um expoente \emph{n}, um elemento primitivo de um
corpo \emph{g} e um número primo \emph{p}, se \(g ^ n = 1\) , então
\(n = 0 mod (p-1)\).

    \hypertarget{aplicauxe7uxe3o-python}{%
\section{Aplicação Python}\label{aplicauxe7uxe3o-python}}

    \hypertarget{imports}{%
\subsection{Imports}\label{imports}}

    Esta secção executa os \emph{imports} de \textbf{Python} que contêm as
funções necessárias para desenvolver a aplicação enunciada.

    \begin{Verbatim}[commandchars=\\\{\}]
{\color{incolor}In [{\color{incolor}1}]:} \PY{k+kn}{import} \PY{n+nn}{os}
        \PY{k+kn}{from} \PY{n+nn}{getpass} \PY{k}{import} \PY{n}{getpass}
        \PY{k+kn}{from} \PY{n+nn}{multiprocessing} \PY{k}{import} \PY{n}{Process}\PY{p}{,}\PY{n}{Pipe}
        \PY{k+kn}{from} \PY{n+nn}{cryptography}\PY{n+nn}{.}\PY{n+nn}{hazmat}\PY{n+nn}{.}\PY{n+nn}{backends} \PY{k}{import} \PY{n}{default\PYZus{}backend}
        \PY{k+kn}{from} \PY{n+nn}{cryptography}\PY{n+nn}{.}\PY{n+nn}{hazmat}\PY{n+nn}{.}\PY{n+nn}{primitives} \PY{k}{import} \PY{n}{hashes}\PY{p}{,} \PY{n}{hmac}
        \PY{k+kn}{from} \PY{n+nn}{cryptography}\PY{n+nn}{.}\PY{n+nn}{hazmat}\PY{n+nn}{.}\PY{n+nn}{primitives}\PY{n+nn}{.}\PY{n+nn}{kdf}\PY{n+nn}{.}\PY{n+nn}{pbkdf2} \PY{k}{import} \PY{n}{PBKDF2HMAC}
        \PY{k+kn}{from} \PY{n+nn}{cryptography}\PY{n+nn}{.}\PY{n+nn}{hazmat}\PY{n+nn}{.}\PY{n+nn}{primitives}\PY{n+nn}{.}\PY{n+nn}{ciphers} \PY{k}{import} \PY{n}{Cipher}\PY{p}{,} \PY{n}{algorithms}\PY{p}{,} \PY{n}{modes}
        \PY{k+kn}{from} \PY{n+nn}{cryptography}\PY{n+nn}{.}\PY{n+nn}{exceptions} \PY{k}{import} \PY{n}{InvalidSignature}
        \PY{k+kn}{from} \PY{n+nn}{base64} \PY{k}{import} \PY{n}{b64encode}\PY{p}{,} \PY{n}{b64decode}
\end{Verbatim}

    \hypertarget{definiuxe7uxe3o-da-classe-de-multiprocessamento}{%
\subsection{Definição da classe de
multiprocessamento}\label{definiuxe7uxe3o-da-classe-de-multiprocessamento}}

    Esta secção tem o propósito de definir a classe de multiprocessamento,
que permite uma comunicação bidireccional com o \emph{Emitter} e o
\emph{Receiver}, sendo estes dois processos criados e implementados pela
\textbf{API}
\href{https://docs.python.org/2/library/multiprocessing.html}{\textbf{multiprocessing}}.

    \begin{Verbatim}[commandchars=\\\{\}]
{\color{incolor}In [{\color{incolor}2}]:} \PY{k}{class} \PY{n+nc}{BiConnection}\PY{p}{(}\PY{n+nb}{object}\PY{p}{)}\PY{p}{:}
            \PY{k}{def} \PY{n+nf}{\PYZus{}\PYZus{}init\PYZus{}\PYZus{}}\PY{p}{(}\PY{n+nb+bp}{self}\PY{p}{,}\PY{n}{left}\PY{p}{,}\PY{n}{right}\PY{p}{)}\PY{p}{:}
                \PY{n}{left\PYZus{}side}\PY{p}{,}\PY{n}{right\PYZus{}side} \PY{o}{=} \PY{n}{Pipe}\PY{p}{(}\PY{p}{)}
                \PY{n+nb+bp}{self}\PY{o}{.}\PY{n}{timeout} \PY{o}{=} \PY{k+kc}{None}
                \PY{n+nb+bp}{self}\PY{o}{.}\PY{n}{left\PYZus{}process} \PY{o}{=} \PY{n}{Process}\PY{p}{(}\PY{n}{target}\PY{o}{=}\PY{n}{left}\PY{p}{,}\PY{n}{args}\PY{o}{=}\PY{p}{(}\PY{n}{left\PYZus{}side}\PY{p}{,}\PY{p}{)}\PY{p}{)}
                \PY{n+nb+bp}{self}\PY{o}{.}\PY{n}{right\PYZus{}process} \PY{o}{=} \PY{n}{Process}\PY{p}{(}\PY{n}{target}\PY{o}{=}\PY{n}{right}\PY{p}{,}\PY{n}{args}\PY{o}{=}\PY{p}{(}\PY{n}{right\PYZus{}side}\PY{p}{,}\PY{p}{)}\PY{p}{)}
                \PY{n+nb+bp}{self}\PY{o}{.}\PY{n}{left} \PY{o}{=} \PY{k}{lambda} \PY{p}{:} \PY{n}{left}\PY{p}{(}\PY{n}{left\PYZus{}side}\PY{p}{)}
                \PY{n+nb+bp}{self}\PY{o}{.}\PY{n}{right} \PY{o}{=} \PY{k}{lambda} \PY{p}{:} \PY{n}{right}\PY{p}{(}\PY{n}{right\PYZus{}side}\PY{p}{)}
            
            \PY{c+c1}{\PYZsh{} Execução manual apenas devido ao facto de a password}
            \PY{c+c1}{\PYZsh{} ter que ser lida em ambos os lados do Pipe}            
            \PY{k}{def} \PY{n+nf}{manual}\PY{p}{(}\PY{n+nb+bp}{self}\PY{p}{)}\PY{p}{:}
                \PY{n+nb+bp}{self}\PY{o}{.}\PY{n}{left}\PY{p}{(}\PY{p}{)}
                \PY{n+nb+bp}{self}\PY{o}{.}\PY{n}{right}\PY{p}{(}\PY{p}{)}
            
\end{Verbatim}

    \hypertarget{definiuxe7uxe3o-do-emissor-e-do-recetor}{%
\subsection{Definição do Emissor e do
Recetor}\label{definiuxe7uxe3o-do-emissor-e-do-recetor}}

    Nesta secção encontram-se implementados os comportamentos do emissor e
do recetor que participam na comunicação bidireccional da aplicação.

    \begin{Verbatim}[commandchars=\\\{\}]
{\color{incolor}In [{\color{incolor}3}]:} \PY{k}{def} \PY{n+nf}{Emissor}\PY{p}{(}\PY{n}{connection}\PY{p}{)}\PY{p}{:}
            \PY{n}{con\PYZus{}salt} \PY{o}{=} \PY{n}{os}\PY{o}{.}\PY{n}{urandom}\PY{p}{(}\PY{l+m+mi}{16}\PY{p}{)}
            \PY{n}{passwd} \PY{o}{=} \PY{n+nb}{bytes}\PY{p}{(}\PY{n}{getpass}\PY{p}{(}\PY{l+s+s1}{\PYZsq{}}\PY{l+s+s1}{Password do emissor: }\PY{l+s+s1}{\PYZsq{}}\PY{p}{)}\PY{p}{,}\PY{l+s+s1}{\PYZsq{}}\PY{l+s+s1}{utf\PYZhy{}8}\PY{l+s+s1}{\PYZsq{}}\PY{p}{)}
            \PY{n}{text\PYZus{}to\PYZus{}send} \PY{o}{=} \PY{n}{os}\PY{o}{.}\PY{n}{urandom}\PY{p}{(}\PY{l+m+mi}{128}\PY{p}{)}
            \PY{n+nb}{print}\PY{p}{(}\PY{l+s+s1}{\PYZsq{}}\PY{l+s+s1}{Texto a cifrar e enviar:}\PY{l+s+s1}{\PYZsq{}}\PY{p}{)}
            \PY{n+nb}{print}\PY{p}{(}\PY{n}{b64encode}\PY{p}{(}\PY{n}{text\PYZus{}to\PYZus{}send}\PY{p}{)}\PY{p}{)}
            \PY{k}{try}\PY{p}{:}
                \PY{c+c1}{\PYZsh{}derivar a chave apartir da password}
                \PY{n}{derivation} \PY{o}{=} \PY{n}{PBKDF2HMAC}\PY{p}{(}
                        \PY{n}{algorithm} \PY{o}{=} \PY{n}{hashes}\PY{o}{.}\PY{n}{SHA256}\PY{p}{(}\PY{p}{)}\PY{p}{,}
                        \PY{n}{length} \PY{o}{=} \PY{l+m+mi}{64}\PY{p}{,}
                        \PY{n}{salt} \PY{o}{=} \PY{n}{con\PYZus{}salt}\PY{p}{,}
                        \PY{n}{iterations} \PY{o}{=} \PY{l+m+mi}{100000}\PY{p}{,}
                        \PY{n}{backend} \PY{o}{=} \PY{n}{default\PYZus{}backend}\PY{p}{(}\PY{p}{)}
                \PY{p}{)}
                \PY{c+c1}{\PYZsh{}Separar a password para cifragem e autenticação}
                \PY{n}{full\PYZus{}key} \PY{o}{=} \PY{n}{derivation}\PY{o}{.}\PY{n}{derive}\PY{p}{(}\PY{n}{passwd}\PY{p}{)}
                \PY{n}{cript\PYZus{}key} \PY{o}{=} \PY{n}{full\PYZus{}key}\PY{p}{[}\PY{p}{:}\PY{l+m+mi}{32}\PY{p}{]}
                \PY{n}{mac\PYZus{}key} \PY{o}{=} \PY{n}{full\PYZus{}key}\PY{p}{[}\PY{l+m+mi}{32}\PY{p}{:}\PY{p}{]}
                
                \PY{c+c1}{\PYZsh{} Utilizar o AES com um dos modos que o torna numa stream cipher}
                \PY{c+c1}{\PYZsh{} para cifrar o criptograma.}
                \PY{n}{nonce} \PY{o}{=} \PY{n}{os}\PY{o}{.}\PY{n}{urandom}\PY{p}{(}\PY{l+m+mi}{16}\PY{p}{)}
                \PY{n}{cipher} \PY{o}{=} \PY{n}{Cipher}\PY{p}{(}\PY{n}{algorithm} \PY{o}{=} \PY{n}{algorithms}\PY{o}{.}\PY{n}{AES}\PY{p}{(}\PY{n}{cript\PYZus{}key}\PY{p}{)}\PY{p}{,}\PY{n}{mode}\PY{o}{=} \PY{n}{modes}\PY{o}{.}\PY{n}{CTR}\PY{p}{(}\PY{n}{nonce}\PY{p}{)}\PY{p}{,}
                \PY{n}{backend} \PY{o}{=} \PY{n}{default\PYZus{}backend}\PY{p}{(}\PY{p}{)}\PY{p}{)}
                \PY{n}{cryptogram} \PY{o}{=} \PY{n}{cipher}\PY{o}{.}\PY{n}{encryptor}\PY{p}{(}\PY{p}{)}\PY{o}{.}\PY{n}{update}\PY{p}{(}\PY{n}{text\PYZus{}to\PYZus{}send}\PY{p}{)}
                
                \PY{c+c1}{\PYZsh{}geração do código de autenticação do criptograma e dos metadados}
                \PY{n}{message\PYZus{}to\PYZus{}authenticate} \PY{o}{=} \PY{n}{cryptogram} \PY{o}{+} \PY{n}{nonce} \PY{o}{+} \PY{n}{con\PYZus{}salt}
                \PY{n}{hasher} \PY{o}{=} \PY{n}{hmac}\PY{o}{.}\PY{n}{HMAC}\PY{p}{(}\PY{n}{mac\PYZus{}key}\PY{p}{,}\PY{n}{hashes}\PY{o}{.}\PY{n}{SHA256}\PY{p}{(}\PY{p}{)}\PY{p}{,}\PY{n}{default\PYZus{}backend}\PY{p}{(}\PY{p}{)}\PY{p}{)}
                \PY{n}{hasher}\PY{o}{.}\PY{n}{update}\PY{p}{(}\PY{n}{message\PYZus{}to\PYZus{}authenticate}\PY{p}{)}
                \PY{n}{hash\PYZus{}msg} \PY{o}{=} \PY{n}{hasher}\PY{o}{.}\PY{n}{finalize}\PY{p}{(}\PY{p}{)}
                
                \PY{c+c1}{\PYZsh{}geração do código de autenticação da chave}
                \PY{c+c1}{\PYZsh{}esforço computacional é reduzido no recetor caso insira a password errada.}
                \PY{n}{hasher\PYZus{}passwd} \PY{o}{=} \PY{n}{hmac}\PY{o}{.}\PY{n}{HMAC}\PY{p}{(}\PY{n}{mac\PYZus{}key}\PY{p}{,}\PY{n}{hashes}\PY{o}{.}\PY{n}{SHA256}\PY{p}{(}\PY{p}{)}\PY{p}{,}\PY{n}{default\PYZus{}backend}\PY{p}{(}\PY{p}{)}\PY{p}{)}
                \PY{n}{hasher\PYZus{}passwd}\PY{o}{.}\PY{n}{update}\PY{p}{(}\PY{n}{cript\PYZus{}key}\PY{p}{)}
                \PY{n}{hash\PYZus{}pass} \PY{o}{=} \PY{n}{hasher\PYZus{}passwd}\PY{o}{.}\PY{n}{finalize}\PY{p}{(}\PY{p}{)}
                \PY{n}{obj} \PY{o}{=} \PY{p}{\PYZob{}}\PY{l+s+s1}{\PYZsq{}}\PY{l+s+s1}{cryptogram}\PY{l+s+s1}{\PYZsq{}}\PY{p}{:} \PY{n}{cryptogram}\PY{p}{,} \PY{l+s+s1}{\PYZsq{}}\PY{l+s+s1}{mac\PYZus{}code}\PY{l+s+s1}{\PYZsq{}}\PY{p}{:} \PY{n}{hash\PYZus{}msg}\PY{p}{,}\PY{l+s+s1}{\PYZsq{}}\PY{l+s+s1}{pass\PYZus{}code}\PY{l+s+s1}{\PYZsq{}}\PY{p}{:}
                \PY{n}{hash\PYZus{}pass}\PY{p}{,} \PY{l+s+s1}{\PYZsq{}}\PY{l+s+s1}{salt}\PY{l+s+s1}{\PYZsq{}}\PY{p}{:} \PY{n}{con\PYZus{}salt}\PY{p}{,} \PY{l+s+s1}{\PYZsq{}}\PY{l+s+s1}{nonce}\PY{l+s+s1}{\PYZsq{}}\PY{p}{:} \PY{n}{nonce}\PY{p}{\PYZcb{}}
                \PY{n}{connection}\PY{o}{.}\PY{n}{send}\PY{p}{(}\PY{n}{obj}\PY{p}{)}
            \PY{k}{except} \PY{n+ne}{Exception} \PY{k}{as} \PY{n}{e}\PY{p}{:}
                \PY{n+nb}{print}\PY{p}{(}\PY{n}{e}\PY{p}{)}
        \PY{k}{def} \PY{n+nf}{Recetor}\PY{p}{(}\PY{n}{connection}\PY{p}{)}\PY{p}{:}
            \PY{n}{passwd} \PY{o}{=} \PY{n+nb}{bytes}\PY{p}{(}\PY{n}{getpass}\PY{p}{(}\PY{l+s+s1}{\PYZsq{}}\PY{l+s+s1}{Password do recetor: }\PY{l+s+s1}{\PYZsq{}}\PY{p}{)}\PY{p}{,} \PY{l+s+s1}{\PYZsq{}}\PY{l+s+s1}{utf\PYZhy{}8}\PY{l+s+s1}{\PYZsq{}}\PY{p}{)}
            \PY{k}{try}\PY{p}{:}
                \PY{n}{obj} \PY{o}{=} \PY{n}{connection}\PY{o}{.}\PY{n}{recv}\PY{p}{(}\PY{p}{)}
                
                \PY{c+c1}{\PYZsh{}Obter parâmetros no objeto}
                \PY{n}{pass\PYZus{}code} \PY{o}{=} \PY{n}{obj}\PY{p}{[}\PY{l+s+s1}{\PYZsq{}}\PY{l+s+s1}{pass\PYZus{}code}\PY{l+s+s1}{\PYZsq{}}\PY{p}{]}
                \PY{n}{cryptogram} \PY{o}{=} \PY{n}{obj}\PY{p}{[}\PY{l+s+s1}{\PYZsq{}}\PY{l+s+s1}{cryptogram}\PY{l+s+s1}{\PYZsq{}}\PY{p}{]}
                \PY{n}{mac\PYZus{}code} \PY{o}{=} \PY{n}{obj}\PY{p}{[}\PY{l+s+s1}{\PYZsq{}}\PY{l+s+s1}{mac\PYZus{}code}\PY{l+s+s1}{\PYZsq{}}\PY{p}{]}
                \PY{n}{salt} \PY{o}{=} \PY{n}{obj}\PY{p}{[}\PY{l+s+s1}{\PYZsq{}}\PY{l+s+s1}{salt}\PY{l+s+s1}{\PYZsq{}}\PY{p}{]}
                \PY{n}{nonce} \PY{o}{=} \PY{n}{obj}\PY{p}{[}\PY{l+s+s1}{\PYZsq{}}\PY{l+s+s1}{nonce}\PY{l+s+s1}{\PYZsq{}}\PY{p}{]}
                
                \PY{c+c1}{\PYZsh{}derivar a chave apartir da password lida}
                \PY{n}{derivation} \PY{o}{=} \PY{n}{PBKDF2HMAC}\PY{p}{(}
                        \PY{n}{algorithm} \PY{o}{=} \PY{n}{hashes}\PY{o}{.}\PY{n}{SHA256}\PY{p}{(}\PY{p}{)}\PY{p}{,}
                        \PY{n}{length} \PY{o}{=} \PY{l+m+mi}{64}\PY{p}{,}
                        \PY{n}{salt} \PY{o}{=} \PY{n}{salt}\PY{p}{,}
                        \PY{n}{iterations} \PY{o}{=} \PY{l+m+mi}{100000}\PY{p}{,}
                        \PY{n}{backend} \PY{o}{=} \PY{n}{default\PYZus{}backend}\PY{p}{(}\PY{p}{)}
                \PY{p}{)}
                \PY{c+c1}{\PYZsh{}Separar a password para cifragem e autenticação}
                \PY{n}{full\PYZus{}key} \PY{o}{=} \PY{n}{derivation}\PY{o}{.}\PY{n}{derive}\PY{p}{(}\PY{n}{passwd}\PY{p}{)}
                \PY{n}{cript\PYZus{}key} \PY{o}{=} \PY{n}{full\PYZus{}key}\PY{p}{[}\PY{p}{:}\PY{l+m+mi}{32}\PY{p}{]}
                \PY{n}{mac\PYZus{}key} \PY{o}{=} \PY{n}{full\PYZus{}key}\PY{p}{[}\PY{l+m+mi}{32}\PY{p}{:}\PY{p}{]}
                
                \PY{c+c1}{\PYZsh{}Autenticação prévia da chave}
                \PY{n}{hasher\PYZus{}passwd} \PY{o}{=} \PY{n}{hmac}\PY{o}{.}\PY{n}{HMAC}\PY{p}{(}\PY{n}{mac\PYZus{}key}\PY{p}{,}\PY{n}{hashes}\PY{o}{.}\PY{n}{SHA256}\PY{p}{(}\PY{p}{)}\PY{p}{,}\PY{n}{default\PYZus{}backend}\PY{p}{(}\PY{p}{)}\PY{p}{)}
                \PY{n}{hasher\PYZus{}passwd}\PY{o}{.}\PY{n}{update}\PY{p}{(}\PY{n}{cript\PYZus{}key}\PY{p}{)}
                \PY{n}{hasher\PYZus{}passwd}\PY{o}{.}\PY{n}{verify}\PY{p}{(}\PY{n}{pass\PYZus{}code}\PY{p}{)}
                
                \PY{c+c1}{\PYZsh{}Autenticação do criptograma e metadados}
                \PY{n}{message\PYZus{}to\PYZus{}authenticate} \PY{o}{=} \PY{n}{cryptogram} \PY{o}{+} \PY{n}{nonce} \PY{o}{+} \PY{n}{salt}
                \PY{n}{hash\PYZus{}msg} \PY{o}{=} \PY{n}{hmac}\PY{o}{.}\PY{n}{HMAC}\PY{p}{(}\PY{n}{mac\PYZus{}key}\PY{p}{,}\PY{n}{hashes}\PY{o}{.}\PY{n}{SHA256}\PY{p}{(}\PY{p}{)}\PY{p}{,}\PY{n}{default\PYZus{}backend}\PY{p}{(}\PY{p}{)}\PY{p}{)}
                \PY{n}{hash\PYZus{}msg}\PY{o}{.}\PY{n}{update}\PY{p}{(}\PY{n}{message\PYZus{}to\PYZus{}authenticate}\PY{p}{)}
                \PY{k}{try}\PY{p}{:}
                    \PY{n}{hash\PYZus{}msg}\PY{o}{.}\PY{n}{verify}\PY{p}{(}\PY{n}{mac\PYZus{}code}\PY{p}{)}
                    
                    \PY{c+c1}{\PYZsh{}Decifrar criptograma}
                    \PY{n}{cipher} \PY{o}{=} \PY{n}{Cipher}\PY{p}{(}\PY{n}{algorithm} \PY{o}{=} \PY{n}{algorithms}\PY{o}{.}\PY{n}{AES}\PY{p}{(}\PY{n}{cript\PYZus{}key}\PY{p}{)}\PY{p}{,}
                    \PY{n}{mode} \PY{o}{=} \PY{n}{modes}\PY{o}{.}\PY{n}{CTR}\PY{p}{(}\PY{n}{nonce}\PY{p}{)}\PY{p}{,}\PY{n}{backend} \PY{o}{=} \PY{n}{default\PYZus{}backend}\PY{p}{(}\PY{p}{)}\PY{p}{)}
                    \PY{n}{plain\PYZus{}text} \PY{o}{=} \PY{n}{cipher}\PY{o}{.}\PY{n}{decryptor}\PY{p}{(}\PY{p}{)}\PY{o}{.}\PY{n}{update}\PY{p}{(}\PY{n}{cryptogram}\PY{p}{)}
                    \PY{n+nb}{print}\PY{p}{(}\PY{l+s+s1}{\PYZsq{}}\PY{l+s+s1}{Texto decifrado:}\PY{l+s+s1}{\PYZsq{}}\PY{p}{)}
                    \PY{n+nb}{print}\PY{p}{(}\PY{n}{b64encode}\PY{p}{(}\PY{n}{plain\PYZus{}text}\PY{p}{)}\PY{p}{)}
                \PY{k}{except} \PY{n}{InvalidSignature} \PY{k}{as} \PY{n}{i}\PY{p}{:}
                    \PY{n+nb}{print}\PY{p}{(}\PY{l+s+s2}{\PYZdq{}}\PY{l+s+s2}{Código de autenticação não é válido!}\PY{l+s+s2}{\PYZdq{}}\PY{p}{)}
            \PY{k}{except} \PY{n+ne}{Exception} \PY{k}{as} \PY{n}{e}\PY{p}{:}
                \PY{n+nb}{print}\PY{p}{(}\PY{n}{e}\PY{p}{)}
            \PY{n}{connection}\PY{o}{.}\PY{n}{close}\PY{p}{(}\PY{p}{)}
                
                
\end{Verbatim}

    \hypertarget{iniciauxe7uxe3o-do-processo}{%
\subsection{Iniciação do
processo}\label{iniciauxe7uxe3o-do-processo}}

    Por último, resta apenas criar um objeto de conexão bidireccional
passando-lhe como argumentos o emissor e o recetor definidos e,
finalmente, prosseguindo com a execução de ambos os processos através da
chamada à função manual.

    \begin{Verbatim}[commandchars=\\\{\}]
{\color{incolor}In [{\color{incolor}4}]:} \PY{n}{BiConnection}\PY{p}{(}\PY{n}{Emissor}\PY{p}{,}\PY{n}{Recetor}\PY{p}{)}\PY{o}{.}\PY{n}{manual}\PY{p}{(}\PY{p}{)}
\end{Verbatim}

    \begin{Verbatim}[commandchars=\\\{\}]
Password do emissor: ········
Texto a cifrar e enviar:
b'Bp7wjCNmvFlppA98akTDp/GDIHBqUOWiF7rSNuZRFzKEJ3dkCZV7xGeqJt+Y8XSeR820m0AVgckpMAY
+GRcTP1UPT+8osucvorZ1pN4RcWZCFH1Nburz1wcIZRwFOYHmMYFubBToo/ZwlqSoJvcESi5vjiceWxc='
Password do recetor: ········
Texto decifrado:
b'Bp7wjCNmvFlppA98akTDp/GDIHBqUOWiF7rSNuZRFzKEJ3dkCZV7xGeqJt+Y8XSeR820m0AVgckpMAY
+GRcTP1UPT+8osucvorZ1pN4RcWZCFH1Nburz1wcIZRwFOYHmMYFubBToo/ZwlqSoJvcESi5vjiceWxc='

    \end{Verbatim}

    Através de uma ligera comparação entre o texto antes de ser cifrado e
depois de ser decifrado, pode-se verificar que é exatamente o mesmo.


\hypertarget{aplicauxe7uxe3o-sagemath}{%
\section{Aplicação SageMath}\label{aplicauxe7uxe3o-sagemath}}

    \hypertarget{criauxe7uxe3o-dos-corpos-finitos-primos}{%
\subsection{Criação dos corpos finitos
primos}\label{criauxe7uxe3o-dos-corpos-finitos-primos}}

    Nesta secção são criadas duas listas:

\begin{itemize}
\tightlist
\item
  A primeira (\emph{P}) corresponde à lista de números primos que serão
  analisados.
\item
  A segunda (\emph{GP}) corresponde à lista de corpos finitos primos,
  onde cada elemento corresponde ao corpo finito primo de um dos números
  primos, previamente adicionados à lista \emph{P}.
\end{itemize}

    \begin{Verbatim}[commandchars=\\\{\}]
{\color{incolor}In [{\color{incolor}1}]:} \PY{n}{P} \PY{o}{=} \PY{p}{[}\PY{l+m+mi}{37}\PY{p}{,} \PY{l+m+mi}{163}\PY{p}{,} \PY{l+m+mi}{263}\PY{p}{,} \PY{l+m+mi}{1009}\PY{p}{]}
        \PY{n}{GP} \PY{o}{=} \PY{p}{[}\PY{n}{GF}\PY{p}{(}\PY{n}{p}\PY{p}{)} \PY{k}{for} \PY{n}{p} \PY{o+ow}{in} \PY{n}{P}\PY{p}{]}
\end{Verbatim}

    \hypertarget{definiuxe7uxe3o-iniciais}{%
\subsection{Definição iniciais}\label{definiuxe7uxe3o-iniciais}}

    De seguida, definiu-se a função que será aplicada aos pontos dos corpos
finitos primos. Esta recebe um ponto \emph{x} e um primo \emph{p} e
calcula ``\emph{x elevado a p menos 2}''. Além disso, é definida também
uma lista de listas \emph{L}, onde cada uma das mesmas contém os pontos
resultantes da aplicação da função aos pontos de um dos corpos finitos
primos.

    \begin{Verbatim}[commandchars=\\\{\}]
{\color{incolor}In [{\color{incolor}2}]:} \PY{k}{def} \PY{n+nf}{f}\PY{p}{(}\PY{n}{x}\PY{p}{,}\PY{n}{p}\PY{p}{)}\PY{p}{:}
            \PY{k}{return} \PY{n}{x}\PY{o}{\PYZca{}}\PY{p}{(}\PY{n}{p}\PY{o}{\PYZhy{}}\PY{l+m+mi}{2}\PY{p}{)}
        
        \PY{n}{L} \PY{o}{=} \PY{p}{[}\PY{p}{[}\PY{n}{f}\PY{p}{(}\PY{n}{x}\PY{p}{,}\PY{n}{p}\PY{p}{)} \PY{k}{for} \PY{n}{x} \PY{o+ow}{in} \PY{n}{GF}\PY{p}{(}\PY{n}{p}\PY{p}{)}\PY{p}{]} \PY{k}{for} \PY{n}{p} \PY{o+ow}{in} \PY{n}{P}\PY{p}{]}
\end{Verbatim}

    \hypertarget{plot-da-funuxe7uxe3o-aplicada-a-cada-um-dos-corpos-finitos}{%
\subsection{Plot da função aplicada a cada um dos corpos
finitos}\label{plot-da-funuxe7uxe3o-aplicada-a-cada-um-dos-corpos-finitos}}

    Esta secção tem como objetivo fazer o \textbf{plot} dos pontos
resultantes da aplicação da função definida anteriormente como
\texttt{f(x,p)} a cada um dos corpos finitos primos, sendo que o
resultado deverá ser apresentado como um gráfico por cada lista de
pontos do corpo finito respetivo.

    \begin{Verbatim}[commandchars=\\\{\}]
{\color{incolor}In [{\color{incolor}3}]:} \PY{c+c1}{\PYZsh{} Plot dos pontos resultantes da aplicação da função aos pontos do}
{\color{incolor}In [{\color{incolor}3}]:} \PY{c+c1}{\PYZsh{} corpo finito primo GF(37)}
        \PY{n}{list\PYZus{}plot}\PY{p}{(}\PY{n}{L}\PY{p}{[}\PY{l+m+mi}{0}\PY{p}{]}\PY{p}{)}
\end{Verbatim}
\texttt{\color{outcolor}Out[{\color{outcolor}3}]:}
    
    \begin{center}
    \adjustimage{max size={0.9\linewidth}{0.9\paperheight}}{output_9_0.png}
    \end{center}
    { \hspace*{\fill} \\}
    

    \begin{Verbatim}[commandchars=\\\{\}]
{\color{incolor}In [{\color{incolor}4}]:} \PY{c+c1}{\PYZsh{} Plot dos pontos resultantes da aplicação da função aos pontos do}
{\color{incolor}In [{\color{incolor}4}]:} \PY{c+c1}{\PYZsh{} corpo finito primo GF(163)}
        \PY{n}{list\PYZus{}plot}\PY{p}{(}\PY{n}{L}\PY{p}{[}\PY{l+m+mi}{1}\PY{p}{]}\PY{p}{)}
\end{Verbatim}
\texttt{\color{outcolor}Out[{\color{outcolor}4}]:}
    
    \begin{center}
    \adjustimage{max size={0.9\linewidth}{0.9\paperheight}}{output_10_0.png}
    \end{center}
    { \hspace*{\fill} \\}
    

    \begin{Verbatim}[commandchars=\\\{\}]
{\color{incolor}In [{\color{incolor}5}]:} \PY{c+c1}{\PYZsh{} Plot dos pontos resultantes da aplicação da função aos pontos do}
{\color{incolor}In [{\color{incolor}5}]:} \PY{c+c1}{\PYZsh{} corpo finito primo GF(263)}
        \PY{n}{list\PYZus{}plot}\PY{p}{(}\PY{n}{L}\PY{p}{[}\PY{l+m+mi}{2}\PY{p}{]}\PY{p}{)}
\end{Verbatim}
\texttt{\color{outcolor}Out[{\color{outcolor}5}]:}
    
    \begin{center}
    \adjustimage{max size={0.9\linewidth}{0.9\paperheight}}{output_11_0.png}
    \end{center}
    { \hspace*{\fill} \\}
    

    \begin{Verbatim}[commandchars=\\\{\}]
{\color{incolor}In [{\color{incolor}6}]:} \PY{c+c1}{\PYZsh{} Plot dos pontos resultantes da aplicação da função aos pontos do}
{\color{incolor}In [{\color{incolor}6}]:} \PY{c+c1}{\PYZsh{} corpo finito primo GF(1009)}
        \PY{n}{list\PYZus{}plot}\PY{p}{(}\PY{n}{L}\PY{p}{[}\PY{l+m+mi}{3}\PY{p}{]}\PY{p}{)}
\end{Verbatim}
\texttt{\color{outcolor}Out[{\color{outcolor}6}]:}
    
    \begin{center}
    \adjustimage{max size={0.9\linewidth}{0.9\paperheight}}{output_12_0.png}
    \end{center}
    { \hspace*{\fill} \\}
    

    A partir da análise dos gráficos previamente gerados podemos verificar
que, em todos eles, à primeira vista, parece existir uma certa
aleatoriedade nos pontos desenhados mas, analisando melhor cada um
deles, podemos confirmar a existência de um padrão simétrico que existe
no gráfico.

    \hypertarget{funuxe7uxf5es-da-proposiuxe7uxe3o-e-auxiliares}{%
\subsection{Funções da proposição e
auxiliares}\label{funuxe7uxf5es-da-proposiuxe7uxe3o-e-auxiliares}}

    Resta agora definir duas funções que são utilizadas para garantir a
proposição na totalidade. A função \texttt{checker(n,g,p)} testa, com o
auxílio das funções anteriormente referidas, a veracidade da proposição
para um dado expoente, elemento primitivo e número primo, ao tentar
encontrar, por amostragem, elementos que provem que a proposição é
incorrecta, seguindo o seguinte algoritmo:

\begin{itemize}
\tightlist
\item
  Se $g^{n} = 1$:

  \begin{itemize}
  \tightlist
  \item
    Se \(n = 0 mod (p-1)\), a proposição verifica-se, pelo que o
    resultado retornado é 0 (não existe erro a ser somado).
  \item
    Se \(n != 0 mod (p-1)\), a proposição falha, pelo que o resultado
    retornado é 1 (proposição é falsa para estes elementos, erro deve
    ser somado).
  \end{itemize}
\item
  Se $g^{n} != 1$, a proposição não é testada.
\end{itemize}

    \begin{Verbatim}[commandchars=\\\{\}]
{\color{incolor}In [{\color{incolor}7}]:} \PY{k}{def} \PY{n+nf}{prop1}\PY{p}{(}\PY{n}{g}\PY{p}{,}\PY{n}{n}\PY{p}{)}\PY{p}{:} 
            \PY{k}{return} \PY{n}{g}\PY{o}{\PYZca{}}\PY{n}{n}
        
        \PY{k}{def} \PY{n+nf}{prop2}\PY{p}{(}\PY{n}{p}\PY{p}{)}\PY{p}{:} 
            \PY{k}{return} \PY{n}{Mod}\PY{p}{(}\PY{l+m+mi}{0}\PY{p}{,}\PY{n}{p}\PY{o}{\PYZhy{}}\PY{l+m+mi}{1}\PY{p}{)}
        
        \PY{k}{def} \PY{n+nf}{checker}\PY{p}{(}\PY{n}{n}\PY{p}{,}\PY{n}{g}\PY{p}{,}\PY{n}{p}\PY{p}{)}\PY{p}{:}
            \PY{k}{if}\PY{p}{(}\PY{n}{prop1}\PY{p}{(}\PY{n}{g}\PY{p}{,}\PY{n}{n}\PY{p}{)} \PY{o}{==} \PY{l+m+mi}{1}\PY{p}{)}\PY{p}{:}
                \PY{k}{if}\PY{p}{(}\PY{n}{prop2}\PY{p}{(}\PY{n}{p}\PY{p}{)} \PY{o}{==} \PY{n}{n}\PY{p}{)}\PY{p}{:}
                    \PY{k}{return} \PY{l+m+mi}{0}
                \PY{k}{else}\PY{p}{:} \PY{k}{return} \PY{l+m+mi}{1}
            \PY{k}{else}\PY{p}{:} \PY{k}{return} \PY{l+m+mi}{0}
\end{Verbatim}

    \hypertarget{criauxe7uxe3o-da-lista-de-expoentes-aleatuxf3rios}{%
\subsection{Criação da lista de expoentes
aleatórios}\label{criauxe7uxe3o-da-lista-de-expoentes-aleatuxf3rios}}

    Por último criou-se uma lista de expoentes, por amostragem, para que
seja averiguada a veracidade da proposição enunciada. São criados 100000
expoentes aleatórios que variam entre 1 e 1 bilião.

    \begin{Verbatim}[commandchars=\\\{\}]
{\color{incolor}In [{\color{incolor}8}]:} \PY{k+kn}{import} \PY{n+nn}{numpy}
        \PY{n}{exponents\PYZus{}list} \PY{o}{=} \PY{p}{[}\PY{n}{numpy}\PY{o}{.}\PY{n}{random}\PY{o}{.}\PY{n}{randint}\PY{p}{(}\PY{l+m+mi}{1}\PY{p}{,}\PY{l+m+mi}{1000000000000}\PY{p}{)} \PY{k}{for} \PY{n}{i} \PY{o+ow}{in} \PY{n+nb}{xrange}\PY{p}{(}\PY{l+m+mi}{100000}\PY{p}{)}\PY{p}{]}
\end{Verbatim}

    \hypertarget{prova-da-proposiuxe7uxe3o-por-amostragem}{%
\subsection{Prova da proposição por
amostragem}\label{prova-da-proposiuxe7uxe3o-por-amostragem}}

    Nesta secção pretendemos, com o auxílio da função
\texttt{checker(n,g,p)} definida anteriormente, verificar, para todos os
corpos finitos primos criados, que a proposição enunciada se verifica,
seguindo o seguinte algoritmo:

\begin{itemize}
\tightlist
\item
  Para todo o número primo \emph{p} em \emph{P}, onde \emph{P} é a lista
  dos números primos utilizados para criar os corpos finitos:

  \begin{itemize}
  \tightlist
  \item
    Para todo o expoente \emph{n} na lista de expoentes previamente
    calculada:

    \begin{itemize}
    \tightlist
    \item
      Calcular o resultado de
      \texttt{checker(n,GF(p).primitive\_element(),p)}, onde o segundo
      argumento é o elemento primitivo do corpo finito do primo
      \emph{p}.
    \item
      Se a proposição se verificar falsa, a variável \emph{checkFalses}
      será incrementada de uma unidade, caso contrário será incrementada
      de 0 (sem falsos).
    \end{itemize}
  \item
    Imprimir o número primo e o número de falsos encontrados na
    aplicação da proposição ao mesmo.
  \end{itemize}
\end{itemize}

    \begin{Verbatim}[commandchars=\\\{\}]
{\color{incolor}In [{\color{incolor}9}]:} \PY{n}{checkFalses} \PY{o}{=} \PY{l+m+mi}{0}
        \PY{k}{for} \PY{n}{p} \PY{o+ow}{in} \PY{n}{P}\PY{p}{:}
            \PY{k}{for} \PY{n}{n} \PY{o+ow}{in} \PY{n}{exponents\PYZus{}list}\PY{p}{:}
                \PY{n}{checkFalses} \PY{o}{+}\PY{o}{=} \PY{n}{checker}\PY{p}{(}\PY{n}{n}\PY{p}{,}\PY{n}{GF}\PY{p}{(}\PY{n}{p}\PY{p}{)}\PY{o}{.}\PY{n}{primitive\PYZus{}element}\PY{p}{(}\PY{p}{)}\PY{p}{,}\PY{n}{p}\PY{p}{)}
            \PY{k}{print}\PY{p}{(}\PY{l+s+s1}{\PYZsq{}}\PY{l+s+s1}{Primo:}\PY{l+s+s1}{\PYZsq{}}\PY{p}{)}
            \PY{k}{print}\PY{p}{(}\PY{n}{p}\PY{p}{)}
            \PY{k}{print}\PY{p}{(}\PY{l+s+s1}{\PYZsq{}}\PY{l+s+s1}{Nº de erros na proposição:}\PY{l+s+s1}{\PYZsq{}}\PY{p}{)}
            \PY{k}{print}\PY{p}{(}\PY{n}{checkFalses}\PY{p}{)}
            \PY{k}{print}\PY{p}{(}\PY{l+s+s1}{\PYZsq{}}\PY{l+s+se}{\PYZbs{}n}\PY{l+s+s1}{\PYZsq{}}\PY{p}{)}
            \PY{n}{checkFalses} \PY{o}{=} \PY{l+m+mi}{0}
\end{Verbatim}

    \begin{Verbatim}[commandchars=\\\{\}]
Primo:
37
Nº de erros na proposição:
0


Primo:
163
Nº de erros na proposição:
0


Primo:
263
Nº de erros na proposição:
0


Primo:
1009
Nº de erros na proposição:
0



    \end{Verbatim}

    Como é possível verificar pelo output produzido, a proposição enunciada
verifica-se para todos os expoentes gerados por amostragem e para todos
os corpos finitos primos criados.

    \hypertarget{conclusuxe3o}{%
\section{Conclusão}\label{conclusuxe3o}}

    Os resultados da resolução deste trabalho prático são, na nossa opinião,
bastante satisfatórios, tendo em conta que os mesmos são os que eram
esperados. O desenvolvimento das aplicações foi feito de um modo gradual
(texto de explicação pelo entre o código), de forma a tornar a leitura e
compreensão do trabalho mais agradável.

As maiores dificuldades que surgiram durante a resolução deste trabalho
prático resumiram-se, na sua maior parte, aos aspetos que dizem respeito
ao desenvolvimento utilizando \textbf{sagemath}, visto que este foi o
primeiro contacto do grupo com esta tecnologia. Além disso, este é
também o primeiro relatório que o grupo produz neste formato, ou seja,
utilizando apenas o \textbf{jupyter} para a produção do código e do
próprio texto do mesmo, pelo que tentámos familiarizar-nos apenas com a
ferramenta, especificamente, na produção do ficheiro necessário, que
representa o relatório na sua totalidade.

    \hypertarget{referuxeancias}{%
\subsection{Referências}\label{referuxeancias}}

    \begin{enumerate}
\def\labelenumi{\arabic{enumi}.}
\tightlist
\item
  \href{https://www.dropbox.com/sh/f0j9adiaw4v3deb/AACBpI2YqgkN5iuVEas5P8wVa/WorkSheets/TP0?dl=0\&subfolder_nav_tracking=1}{Worksheets
  do TP0 fornecidas pelo professor}
\item
  \href{https://cryptography.io/en/latest/hazmat/primitives/symmetric-encryption/}{Cryptography
  - Symmetric encryption}
\item
  \href{https://cryptography.io/en/latest/hazmat/primitives/mac/}{Cryptography
  - Message Authentication Codes}
\item
  \href{https://cryptography.io/en/latest/hazmat/primitives/key-derivation-functions/}{Cryptography
  - Key Derivation Functions}
\item
  \href{http://doc.sagemath.org/html/en/reference/finite_rings/sage/rings/finite_rings/finite_field_prime_modn.html}{SageMath
  - Finite Prime Fields}
\item
  \href{http://doc.sagemath.org/html/en/reference/plotting/sage/plot/plot.html}{SageMath
  - 2D Plotting}
\item
  \href{http://doc.sagemath.org/html/en/reference/finite_rings/sage/rings/finite_rings/finite_field_base.html}{SageMath
  - Base Classes for Finite Fields}
\end{enumerate}


    % Add a bibliography block to the postdoc
    
    
    
    \end{document}